\documentclass[12pts]{article}
\title{\textbf{NATIONAL INSTITUTE OF TECHNOLOGY, RAIPUR(C.G)}}
\usepackage{graphicx}
\usepackage[document]{ragged2e}
\graphicspath{{Images/}}
\usepackage{calc}
\usepackage{eso-pic}

\newlength{\PageFrameTopMargin}
\newlength{\PageFrameBottomMargin}
\newlength{\PageFrameLeftMargin}
\newlength{\PageFrameRightMargin}

\setlength{\PageFrameTopMargin}{1cm}
\setlength{\PageFrameBottomMargin}{1cm}
\setlength{\PageFrameLeftMargin}{1cm}
\setlength{\PageFrameRightMargin}{1cm}

\makeatletter

\newlength{\Page@FrameHeight}
\newlength{\Page@FrameWidth}

\AddToShipoutPicture{
  \thinlines
  \setlength{\Page@FrameHeight}{\paperheight-\PageFrameTopMargin-\PageFrameBottomMargin}
  \setlength{\Page@FrameWidth}{\paperwidth-\PageFrameLeftMargin-\PageFrameRightMargin}
  \put(\strip@pt\PageFrameLeftMargin,\strip@pt\PageFrameTopMargin){
    \framebox(\strip@pt\Page@FrameWidth, \strip@pt\Page@FrameHeight){}}}

\makeatother


\author{\textbf{SACHIN KUMAR}\\sachin.1107sk@gmail.com\\Roll No: 21111047}
\date{September 9, 2022}
\begin{document}
\maketitle
\begin{figure}[h]
\centering
\includegraphics[scale=0.92]{aaaa.jpg}
\caption{National Institute of Technology, Raipur}
\end{figure}
\centering
\textbf{ASSIGNMENT-1 OF \underline{ANATOMY AND PHYSIOLOGY}}\\
\centering
UNDER THE SUPERVISION OF \underline{DR. SAURABH GUPTA SIR}\\

\clearpage

\section*{What is anatomy and physiology?}
\begin{large}
\begin{flushleft}
Human anatomy and physiology is the study of the structure and function of the human body. The subject of anatomy focuses on the intricate physical structure of the body and its various systems, from the microscopic level of molecules and cells to the organs themselves. The subject of physiology focuses on the workings, or interactions, of the body's structures. Like anatomy, physiological studies also range from the microscopic level to the larger area of entire organ systems, like the cardiac and respiratory systems, and the biochemical processes that occur within these systems.
\end{flushleft}
\end{large}

\begin{figure}[h]
\centering
\includegraphics[scale=0.22]{ana.jpg}
\end{figure}

\section*{Levels of Structural 
Organization and 
Body Systems}

\begin{flushleft}
The human body has 6 main levels of structural organization. We will begin this lesson with the simplest level within the structural hierarchy.\\

\textbf{\underline{Chemical level}}– includes the tiniest building blocks of matter, atoms, which combine to form molecules, like water. In turn, molecules combine to form organelles, the internal organs of a cell.\\
\textbf{\underline{Cellular level}}– the cellular level is made up of the smallest unit of living matter, the cell with specific tasks and functions.\\

\textbf{\underline{Tissue Level}}– groups of similar cells that have a common function. A tissue must contain two different types of cells. Types include epithelium, connective, muscle, and nervous tissue.\\ 

\textbf{\underline{Organ Level}}– an organ is a structure composed of at least two different tissue types that perform a specific function within the body. Examples include the brain,stomach,etc.\\

\textbf{\underline{System Level}}– One or more organs work in unitedly for functioning. For instance, the heart and blood vessels work together and circulate blood throughout the body to provide oxygen and nutrients to cells. Cardiovascular system integumentary, skeletal, nervous, muscular, endocrine, respiratory, lymphatic, digestive, urinary, and reproductive systems are some organ system.

\textbf{\underline{Organism Level}}– Sum of all structural level which performs desired actions for ex. it is the human being (or organism) as a whole.
\end{flushleft}

\section*{Basic Life Processes}
\begin{flushleft}
\begin{itemize}
\item \textbf{Metabolism} - Sum of all chemical processes
\item \textbf{Responsiveness}- body’s ability to detect and respond to 
changes
\item \textbf{Movement} - Motion of the whole body even at atomic level
\item \textbf{Growth}- Increment or decrement in body size
\item \textbf{Differentiation} - Development of a cell 
from an unspecialized to a specialized state
\item \textbf{Reproduction} -For the production of new Individual
\end{itemize}
\end{flushleft}

\section*{Homeostasis}
\begin{flushleft}
is the maintenance of relatively stable conditions in the body’s internal environment.
\subsection*{Homeostasis and Body fluids:}
An important aspect of homeostasis is maintaining the volume and composition of body fluids.
\begin{itemize}
\item Intracellular fluid(ICF) - fluid within cells
\item Extracellular fluid(ECF) - fluid outside body cells
\item Interstitial fluid(IF) - The ECF that fills the narrow  spaces between cells
\end{itemize}

\subsection*{Control of Homeostasis:}
Homeostasis in the human body is continually being disturbed. Some 
disruptions come from the external environment in the form of physical insults such as the intense heat of a hot summer day or a lack of 
enough oxygen for that two-mile run. Other disruptions originate in 
the internal environment, such as a blood glucose level that falls too 
low when you skip breakfast. Homeostatic imbalances may also occur 
due to psychological stresses in our social environment.

\subsection*{Feedback Systems}
A feedback system  is a cycle of events in which the status of a body condition is monitored, evaluated, changed, re-monitored, re-evaluated,and so on.
Any disruption that changes a controlled 
condition is called a stimulus.A feedback system includes three basic 
components: a receptor, a control center, and an effector...

\begin{figure}[h]
\centering
\includegraphics[scale=0.5]{feedback.jpg}
\end{figure}
\begin{enumerate}
\item \textbf{Receptor} - sends input to a control center
\item \textbf{Control center} - evaluates the input it receives from receptors, and 
generates output commands when they are needed.Output from 
the control center typically occurs as nerve impulses, or hormones or other chemical hormones.This pathway is termed as efferent pathway.
\item \textbf{an Effector}-  is a body structure that receives output from the control center and produces a response or effect that changes the controlled condition.\\
\end{enumerate}
Here we are having two type of feedback systems.
\begin{itemize}
\item Positive Feedback System
\item Negative Feedback System
\end{itemize}
\end{flushleft}
\vspace{10 mm}
\centering
\begin{LARGE}
\underline{THANK YOU}
\end{LARGE}
\end{document}